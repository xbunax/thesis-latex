
\section{总结与展望}
\subsection{总结}
首先我们介绍了\ce{Ce3(Pd/Pt)In11},这种重费米子材料可以允许反铁磁和非常规超导的共存,并且两者还有竞争与共存机制,通过对这类材料的研究能够加深我们对重费米子材料超导机制的理解。并且我们对Hubbard模型近似,通过设置合适的参数,利用DQMC对周期安德森模型进行了模拟,与文献的结果符合较好。并且初步完成了对Hubbard模型的计算模拟,初步验证了在\ce{Ce3(Pd/Pt)In11}中超导和反铁磁的机制。

同时我们我们利用python对计算结果进行处理,成功实现了数据的自动化处理,极大方便了后期的数据备份以及分析。

\subsection{未来展望}
我们计划进一步通过设置参数利用DQMC对Hubbard模型进行进一步模拟,将\ce{Ce3(Pd/Pt)In11}材料中的超导与反铁磁共存与竞争机制通过计算机模拟出来。

对于反铁磁与超导态在微观上是如何相互耦合的,我们目前还没能很好的解释,也是我们未来的研究方向。

同样对于重费米子材料的研究,例如非常规超导的测量以及调控,能够拓展我们对强关联物理的理解,并且由于重费米子材料中的相互作用比较复杂,目前还未形成一个朴实的模型。目前主要有唯象自旋涨落理论、涨落交换近似理论、轨道选择理论、自洽重整化理论等多种理解思路。重费米子材料在当前的科学研究中发挥着重要作用
