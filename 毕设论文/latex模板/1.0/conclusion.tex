
\section{总结与展望}

首先我们介绍了\ce{Ce3(Pd/Pt)In11},这种重费米子材料可以允许反铁磁和非常规超导的共存,并且两者还有竞争与共存机制,通过对这类材料的研究能够加深我们对重费米子材料超导机制的理解。本文初步完成了对我们所构建的模型的计算模拟,以及初步验证了在\ce{Ce3(Pd/Pt)In11}中超导和反铁磁的机制。

同时本文利用python对计算结果进行处理,成功实现了数据的自动化处理,极大方便了后期的数据备份以及分析。


我们计划进一步通过设置参数利用DQMC对我们构建出来的有效模型进行进一步模拟,阐明\ce{Ce3(Pd/Pt)In11}材料中的超导与反铁磁共存与竞争机制。

对于反铁磁与超导态在微观上是如何相互耦合的,本文目前未能很好的解释,也是未来的研究方向。
对于重费米子材料的研究,例如非常规超导的测量以及调控,能够拓展我们对强关联物理的理解。由于重费米子材料中的相互作用比较复杂,目前还未形成一个普适的模型。目前主要有唯象自旋涨落理论、涨落交换近似理论、轨道选择理论、自洽重整化理论等多种理解思路。此外更重要的是,重费米子体系还存在很多潜在的拓扑超导体,比如$\ce{PrOs_4Sb_{12}}$、\ce{URuSi2}、\ce{UPt3}、\ce{UBe3}等,因此重费米材料有很高的研究价值。
