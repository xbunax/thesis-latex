\begin{thebibliography}{20}
\addcontentsline{toc}{section}{参考文献}
\bibitem{shi2012} Kratochvílová M, Prokleška J, Uhlířová K, et al. Coexistence of antiferromagnetism and superconductivity in heavy fermion cerium compound Ce3PdIn11[J]. Scientific reports, 2015, 5(1): 1-11.
\bibitem{MLAM} Wu W, Tremblay A M S. d-wave superconductivity in the frustrated two-dimensional periodic Anderson model[J]. Physical Review X, 2015, 5(1): 011019.
\bibitem{DL} Prokleška J, Kratochvílová M, Uhlířová K, et al. Magnetism, superconductivity, and quantum criticality in the multisite cerium heavy-fermion compound Ce 3 PtIn 11[J]. Physical Review B, 2015, 92(16): 161114.
\bibitem{机器学习实战} Custers J, Diviši M, Kratochvílová M. Quantum critical behavior and superconductivity in new multi-site cerium heavy fermion compound Ce3PtIn11[C]//Journal of Physics: Conference Series. IOP Publishing, 2016, 683(1): 012005.
\bibitem{JMLR} Das D, Gnida D, Bochenek Ł, et al. Magnetic field driven complex phase diagram of antiferromagnetic heavy-fermion superconductor Ce3PtIn11[J]. Scientific Reports, 2018, 8(1): 1-10.
\bibitem{shi20180702} Kambe S, Sakai H, Tokunaga Y, et al. In 115 NQR study with evidence for two magnetic quantum critical points in dual Ce site superconductor Ce 3 PtIn 11[J]. Physical Review B, 2020, 101(8): 081103.
\bibitem{DLWP} Fukazawa H, Kumeda K, Shioda N, et al. Successive magnetic transitions in the heavy-fermion superconductor Ce 3 PtIn 11 studied by In 115 nuclear quadrupole resonance[J]. Physical Review B, 2020, 102(16): 165124.
%\bibitem{DLWPfull} Eli. Stevens and Luca. Antiga, Deep Learning with PyTorch[M]. Manning Publications Co.
\bibitem{vgg16} Shioda N, Kumeda K, Fukazawa H, et al. Determination of the magnetic q vectors in the heavy fermion superconductor Ce 3 PtIn 11[J]. Physical Review B, 2021, 104(24): 245119.
\bibitem{sklearn} 沈斌, 袁辉球. 磁性量子相变[J]. 物理, 2020, 49(9): 570-578.
\end{thebibliography}
