\pagenumbering{arabic}

\section{绪论}
\subsection{研究背景}
重费米行为由K.Andres,J.E.Graebner和H.R.Ott在1975年发现,他们观察到\ce{CeAl_3}中线性比热容
变化很大。在对掺杂超导体的研究中得出结论,一种材料中存在局域磁矩和超导态是不相容的,但是
在1979年,Frank Steglich等人在\ce{CeCu_2Si_2}中发现了重费米子的超导性。H.von Löhneysen等
人于1994年在重费米子化合物的相图中发现了量子临界点和非费米液体行为,导致人们对这些化合物
的研究产生了新的兴趣。另一个实验突破是(由Gil Lonzarich小组证明)重费米子中的量子临界性
可能是非常规超导性的原因。

常规超导现象的微观解释由Bardeen等人于1957年完成即BCS理论。而电阻极小值现象经过30多年的研究,最
终确定与金属杂质的存在由直接关联,并发现电阻在低温下呈现对数增长的行为。1964年,日本物理学家Kondo
借助微扰论方法处理电子和磁性杂质的相互作用,发现自旋反转散射过程会对电阻率产生正比于$-logT$的贡献,
进而与声子散射的$T^5$贡献相结合,可以在理论上解释稀磁合金的电阻极小现象。这一散射过程称为Kondo散射。
在微观图像上,受到超导现象中自旋相反的电子配对形成自旋相反的束缚态,被称为Konde单态。Konde效应
由一个特征温标来刻画,$T_K ~ \rho^{-1}e^{-1/J \rho_0}$,简称Kondo温度,其中J为磁性杂质遇到导带电子自旋的耦合强度,$\rho_0$为费米面上电子态密度。低于$T_K$时,微扰论失效,电阻率不再遵循$-logT$的行为。



\subsection{研究目的与意义}
\subsubsection{研究目的}
重费米子材料在当前的科学研究中发挥着重要作用,但是作为研究非常规超导、非费米液体行为和量子
临界性的原型材料,重费米子化合物中局域磁矩和传导电子之间的实际相互作用仍未完全了解,并且是
正在进行的研究主题。

\subsubsection{研究意义}



\subsection{研究内容与方法}



\subsubsection{研究内容}


\subsubsection{研究方法}


\subsection{研究思路}
